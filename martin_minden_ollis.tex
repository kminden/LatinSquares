\documentclass[12pt,a4paper]{article}
\usepackage{amsfonts,amssymb}
\usepackage{latexsym}
\usepackage[usenames,dvipsnames]{color}
\usepackage{subfigure}
\usepackage{amsmath}
\usepackage{amsthm}
\bibliographystyle{plain}
\usepackage{fullpage}

%environments
\newtheorem{thm}{Theorem}[section]
\newtheorem{exa}[thm]{Example}
\newtheorem{lem}[thm]{Lemma}
\newtheorem{cor}[thm]{Corollary}
\newtheorem{prop}[thm]{Proposition}
\newtheorem{conj}[thm]{Conjecture}
\newtheorem{prob}[thm]{Problem}
\newtheorem{ques}[thm]{Question}


%notation
\newcommand{\Z}{\mathbb{Z}}
\newcommand{\N}{\mathbb{N}}
\newcommand{\R}{\mathbb{R}}
\newcommand{\Q}{\mathbb{Q}}
\newcommand{\F}{\mathbb{F}}
\newcommand{\ub}{\underbrace}
\newcommand{\ep}{\mathop{\rm EP}\nolimits}
\newcommand{\rowrev}{\mathop{\rm rowrev}\nolimits}
\newcommand{\colrev}{\mathop{\rm colrev}\nolimits}
\newcommand{\rot}{\mathop{\rm rot}\nolimits}
\newcommand{\subsoma}{\mathop{\rm subSOMA}\nolimits}
\DeclareMathOperator{\dom}{dom}
\DeclareMathOperator{\ran}{range}
\newcommand{\rest}{\mathbin{\upharpoonright}}
\newcommand{\To}{\longrightarrow}


\newcommand{\st}{\; | \;}
\newcommand{\set}[2]{\left\{#1\st #2 \right\}}
\newcommand{\seq}[2]{\langle #1 \st #2 \rangle}

\renewcommand{\P}{\mathbb{P}}
\renewcommand{\a}{\textup{\textbf{a}}}
\renewcommand{\b}{\textup{\textbf{b}}}
\newcommand{\g}{\textup{\textbf{g}}}
\renewcommand{\c}{\textup{\textbf{c}}}
\renewcommand{\d}{\textup{\textbf{d}}}
\newcommand{\e}{\textup{\textbf{e}}}
\newcommand{\f}{\textup{\textbf{f}}}
\renewcommand{\r}{\overline r}

%\setlength{\parindent}{0in}
%\setlength{\parskip}{3mm}



\begin{document}



\title{Infinite Latin Squares: Neighbor Balance and Orthogonality}

\author{Gage~N.~Martin$^{1}$, Kaethe Minden$^2$ and M.~A.~Ollis$^{2,}$\footnote{Corresponding author, email address: \texttt{matt@marlboro.edu.}}   \\
              \\
              {\it ${}^1$Boston College, Somewhere in Boston, } \\
              {\it Massachusetts 12345, USA.}
              \\
              \\
              {\it ${}^2$Marlboro College, P.O.~Box A, Marlboro,} \\    
              {\it Vermont 05344, USA.}  }
              
%\date{}

\maketitle



\begin{abstract}
Regarding neighbor balance, we consider natural generalizations of $D$-complete Latin squares and Vatican squares from the finite to the infinite.   We show that if~$G$ is an infinite abelian group such that the number of square elements is equinumerous with the whole group then it is possible to permute the rows and columns of the Cayley table to give an infinite Vatican square.  We also construct an Vatican square of every infinite order that is not obtainable by permuting the rows and columns of a Cayley table.  Regarding orthogonality, we show that if~$G$ is as above then $G$ has a set of~$|G|$ mutually orthogonal orthomorphisms and hence there is a set of~$|G|$ mutually orthogonal Latin squares based on~$G$.  \texttt{[And we can maybe do a bit better than this last sentence.]}



\vspace{3mm}
\noindent
{\bf Keywords:} complete Latin square; complete mapping; directed terrace; infinite design; infinite Latin square; mutually orthogonal Latin squares; orthomorphism; R-sequencing; sequencing; Vatican square.
\end{abstract}



\section{Introduction}\label{sec:intro}


A finite Latin square is {\em row complete} or {\em Roman} if  any two distinct symbols appear in adjacent cells within rows once in each order.  If the transpose of a Latin square is row complete then the square is {\em column complete}; a square that is row complete and column complete is {\em complete}.  Finite row complete squares exist for all composite orders~\cite{Higham98} and finite complete squares are known to exist for all even orders~\cite{Gordon61} and many odd composite orders at which a nonabelian group exists; see, for example, \cite{Ollis14}.

Vatican and $D$-complete  squares strengthen this notion of completeness. 
A Latin square is is {\em row D-complete} if any two distinct symbols appear in cells that are distance~$d$ apart in rows at most once in each order for each~$d \leq D$. {\em Column D-completeness} is defined analogously and a square that is both row and column D-complete is {\em D-complete}.  The 1-completeness property is the same as completeness.

An $(n-1)$-complete square of order~$n$ is called {\em Vatican}; that is, Vatican squares have the pair-occurrence restriction at every possible distance.

Vatican squares are known to exist for all orders that are one less than a prime.  In addition to this, 2-complete squares are known to exist at orders~$2p$ where~$p$ is a prime congruent to 5, 7 or 19 modulo~24, orders~$2m$ where~$5 \leq m \leq 25$, and order~21 \cite{TuscanCRC,OllisTFSG}.


In this paper we extend these notions to the infinite and prove various existence results.  As in \cite{CW02}, we use Zermelo-Fraenkel set theory with the axiom of choice. In order to work with arbitrary sizes of infinity, we work with ordinals and cardinals. Ordinals are order types of well-ordered sets. Two well-ordered sets have the same order type if there is an isomorphism between them preserving order. Every natural number is an ordinal, with limit $\omega$, the order type of the natural numbers. Then we can keep going, and obtain $\omega+1, \omega+2$, and so on, reaching the limit of that process, which is $\omega+\omega$ or $\omega\cdot 2$. And so on. We work with transfinite induction. This works exactly like with the natural numbers. Just as the natural numbers are defined recursively, and statements are proved about them by induction, such is the case with ordinals. The only difference is that now we have not just successor ordinals, like natural numbers, but also limit ordinals, like $\omega$ and $\omega\cdot 2$ and $\omega^2$ and $\omega^\omega$ and so on - these ordinals cannot be seen as a successor of any ordinal. Thus with transfinite induction we have a base case, the successor case (inductive step) and also the limit stages. The cardinality of a set $x$ is the least ordinal $\alpha$ bijective with it. Such ordinals are cardinal numbers. 
%The least infinite cardinal is $\aleph_0 = \omega$, this is the size of any countable infinity. The next larger cardinal we denote by $\aleph_1$, the first uncountable cardinal. 

We require a definition of an infinite Latin square that allows us to talk about spatial relationships.  This is accomplished by using a subset of an ordered field to index the rows and columns.  When that field is~$\Q$ or~$\R$, the infinite Latin squares we obtain are natually embedded in~$\R^2$.

Let~$\F$ be an ordered field and let~$I \subseteq \F$.   For each~$d \in \F^+$ let~$I_{(d)} = \set{ i \in I }{  i+d \in I }$.  If $|I| = |\F|$ and for each~$d$ we have either $|I_{(d)}| = |\F|$ or~$|I_{(d)}| = 0$, then~$I$ is an {\em index set}.   For our purposes, we may assume that~$I_{(1)} \neq \emptyset$ without loss of generality.

Given an index set~$I$, a {\em Latin square} on~$I$ with symbol set~$X$ is a function $L: I \times I \rightarrow X$ such that for each~$i \in I$ the restriction of $L$ to $I \times \{i\}$ is a bijection with~$X$, as is the restriction to~$\{i\} \times I$.   In other words, each symbol appears once in each ``row" and once in each ``column."  

This definition is  compatible with the definition for Latin squares of arbitrary cardinality of Hilton and Wojciechowski~\cite{HW05}.  In the countable case with~$I = \N \subseteq \Q$ or $I = \Z \subseteq \Q$ we get the ``quarter-plane Latin squares" and ``full-plane Latin squares" respectively of Caulfield~\cite{Caulfield96}.




The definition of completeness for infinite squares is obtained by identifying the ideas of adjacency and being at distance~1.  This is perfectly natural when~$I \in \{ \N ,\Z\}$ and again matches the definition of Caulfield~\cite{Caulfield96}.   It does not seem to capture a property of particular combinatorial interest otherwise, but when we move to generalizing Vatican squares we get the very natural notion of pairs appearing once at all distances.  Indeed, the definition of an infinite Vatican square is arguably more natural than the finite version as it allows every pair to appear exactly once at every distance rather than merely at most once.

Formally, an infinite Latin square on an index set~$I$  is {\em row complete} or {\em Roman} if each pair of distinct symbols appears exactly once in each order at distance~1 in rows.  The square is {\em complete} if the corresponding property also holds in columns.   An infinite Latin square with indexing set~$I$ is {\em row $D$-complete} if each pair of symbols appear exactly once in each order at distance~$d$ in rows for each~$d$ such that~$I_{(d)} \neq \emptyset$ and $0 < d \leq D$. The square is~{\em $D$-complete} if the corresponding property also holds in columns. Further, the square is {\em Vatican} if for each~$d$ with~$I_{(d)} \neq \emptyset$ we have that each pair of distinct symbols appears at distance~$d$ exactly once in each order in rows and once in each order in columns.



Our first method for constructing squares uses Cayley tables of groups.   In the finite case all known constructions for complete squares--and hence $D$-complete and Vatican squares---use the notion of ``sequenceability" of a group and generalisations of it.  In the next section we show that similar notions are sufficient to construct infinite $D$-complete and Vatican  squares.  Say that an infinite group~$G$ is~{\em squareful} if the set~$\{ g^2 : g \in G\}$ has the same cardinality as~$G$.  If~$G$ is an abelian squareful group and~$I$ is an index set with~$|I| = |G|$, then we can construct an infinite Vatican square using the Cayley table of~$G$. 

In Section~\ref{sec:notgp} we explore non-group-based methods.  We show that there is a Vatican square of each infinite order that cannot be produced from the rows and columns of a Cayley table.  Whether a finite Vatican square with this property exists is an open question.  We also show that there is a Latin square of each infinite order such that with no orderings of its rows and columns gives a Vatican (or even row-complete) square.


As infinite sets can be bijective with proper subsets of themselves, we can define a variation on Vatican squares that only makes sense for infinite orders.
Say that an infinite Latin square on index set~$I$ is {\em semi-Vatican} if for each~$d$ with~$I_{(d)} \neq \emptyset$ we have that each pair of distinct symbols appears at distance~$d$ exactly once in rows and once in columns.  Although this does not have a finite analogue, all known constructions for Vatican squares of finite order~$n$ have $n/2$ rows that together form a ``row semi-Vatican rectangle" and the remaining $n/2$ rows are the reverse of these ones.

All of the results for Vatican squares transfer to the semi-Vatican case with little modification.  In addition to this, looking at the semi-Vatican case allows for an explicit construction of one in the case~$I = \R$ using only the tools of undergraduate Calculus.


Moving to orthogonality, two finite Latin squares on a symbol set~$X$ are orthogonal if for  each pair~$(x_1, x_2) \in X \times X$ there is exactly one position such that~$x_1$ is in that position in the first square and~$x_2$ is in that position in the second pair.  This definition carries over without modification to the infinite case (the countable version of which is given in \cite[p.~116]{DK15}).

In Section~\ref{sec:orth} we see that the methods from Section~\ref{sec:cayley} may be quickly adapted to produce sets of~$\kappa$ mutually orthogonal Latin squares of order~$\kappa$ for all infinite orders~$\kappa$ via Cayley tables of abelian squareful groups.  This is analgous to the finite construction of ``orthomorphisms" via ``R-sequencings".  \texttt{[And we maybe do even more stuff with orthomorphisms]}  


\section{Vatican squares from groups}\label{sec:cayley}


Let~$I$ be an index set in an ordered field~$\F$.  Let~$G$ be a group of order~$|I|$.  For a bijection~${\bf a}: I \rightarrow G$ define a function~${\bf a}_{(d)}: I_{(d)} \rightarrow G \setminus \{ e \}$ for each~$d \in \F^+$ with~$I_{(d)} \neq \emptyset$ by
$${\bf a}_{(d)}(i) = {\bf a}(i)^{-1}{\bf a}(i+d).$$
If there is a~$D$ such that for all~$d < D$ with~$I_{(d)} \neq \emptyset$ we have that each ${\bf a}_{(d)}$ is a bijection, then~${\bf a}$ is a {\em directed $T_D$-terrace for~$G$}.  If ${\bf a}_{(d)}$ is a bijection for all~$d$  with~$I_{(d)} \neq \emptyset$ then ${\bf a}$ is  a {\em directed $T_{\infty}$-terrace} for~$G$.

These definitions closely mimic the versions for finite groups~\cite{Anderson90}.   They can be used to produce Latin squares with neighbor balance properties in much the same way.  Theorem~\ref{th:terrace2square} generalizes Gordon's result~\cite{Gordon61} for finite complete squares and Anderson's~\cite{Anderson90} and Etzion, Golomb and Taylor's results~\cite{EGT89} for finite Vatican squares to the infinite.  

For any bijection~${\bf a}: I \rightarrow G$ define a square~$L({\bf a}) = (\ell_{ij})$ by $\ell_{ij} = a(i)^{-1}a(j)$.   As~${\bf a}$ is a bijection, each row and column contains each symbol exactly once and so~$L$ is a Latin square.  Call a Latin square created in this way {\em based on~$G$}, or simply {\em group-based}.

\begin{thm}\label{th:terrace2square}
Let~$G$ be a group of infinite order~$\kappa$.  If~$G$ has a directed $T_{D}$-terrace for an index set~$I$ then there is a $D$-complete Latin square of order~$|G|$ on~$I$.  Further, if~$G$ has a directed $T_{\infty}$-terrace then there is a Vatican square of order~$|G|$.
\end{thm}

\begin{proof}
Let~${\bf a}$ be a directed~$T_D$-terrace for~$G$ on~$I$ and consider $L({\bf a})$.

Take~$x$ and~$y$ to be distinct elements of~$G$.  As~${\bf a}_{(d)}$ is a bijection, there is a unique~$j$ with ${\bf a}(j)^{-1}{\bf a}(j+d) = x^{-1}y$ and a unqiue~$i$ with ${\bf a}(i)^{-1}{\bf a}(j)=x$.  We therefore have that~$x$ appears in row~$i$ and column~$j$ of~$L({\bf a})$ and that~$y$ appears in row~$i$ and column~$j+d$ of~$L({\bf a})$ and that~$x$ and~$y$ do not appear anywhere else with~$y$ exactly distance~$d$  to the right of~$x$.

There is also a unique~$i$ with $xy^{-1} = {\bf a}(i)^{-1}{\bf a}(i+d)$ and then a unique~$j$ with ${\bf a}(i)^{-1}{\bf a}(j)=x$.  This identifies a unique place where~$y$ appears at exactly distance~$d$ above~$x$ in the square.  Therefore $L({\bf a})$ is a $D$-complete square on~$I$.

If we replace~${\bf a}$ with a directed $T_{\infty}$-terrace in the above argument we see that~$L({\bf a})$ is a Vatican square on~$I$.
\end{proof}

The 1-complete case with~$I \in \{ \N, \Z \}$ of Theorem~\ref{th:terrace2square} is equivalent to results of Caulfield~\cite{Caulfield96}.

We wish to know which infinite groups have directed $T_D$- and~$T_{\infty}$-terraces.  

We use transfinite induction to build such terraces, and later to build the squares more generally. In order to do this we will use posets of the form $\P=\langle \P, \leq\rangle$. Following convention in set theory, we refer to elements of posets as conditions. We will typically start off the transfinite induction with a condition, and then obtain what we will call strengthenings (a strengthening of a condition $q$ is a condition $p$ satisfying $p\leq q$) of this condition to form a chain of approximations to the object we are attempting to build. In order to organize the induction to give us the properties we ultimately want in the union of the chain of conditions, we will build the chain in each stage by meeting (by getting conditions which are inside of) judiciously chosen dense sets. Density here is in the sense of the order topology: a set $D \subseteq \P$ is dense so long as for all $p \in \P$ there is at least one $d \in D$ satisfying $d\leq p$. The union of this chain will then give us the desired square, hopefully. 


\begin{thm}\label{th:T_infty}
Let~$G$ be an abelian squareful group of infinite order.   Then~$G$ has a directed $T_{\infty}$-terrace.
\end{thm}

\begin{proof}
Let $I$ be an index set in an ordered field $\F$, where $|I|=\kappa$ is the order of $G$. We build a $T_\infty$-terrace for $G$ by transfinite induction on $\kappa$. Consider the poset $\P$ consisting of partial directed $T_\infty$-terraces on $G$. Such partial terraces are defined as in the definition of a directed $T_\infty$-terrace, except the functions $\a$ and $\a_{(d)}$ for each $d \in \F^+$ with $I_{(d)} \neq \emptyset$ are only required to be injective partial functions from $I$ to $G$. Here $\P$ should be partially ordered so that $\a \leq \b$ if and only if $\dom \b \subseteq \dom \a$ and $\a \rest \dom \b = \b$.

It is not hard to see that $\P$ is $<\kappa$-closed. Suppose we have an decreasing chain of partial directed $T_\infty$-terraces, $\a_{\alpha}$ for $\alpha < \kappa$, on $G$. Then the union of all of them, $\a$, is itself partial directed $T_\infty$-terrace on $G$. Indeed, $\a$ is an injection since each $\a_\alpha$ in the chain is. For each $d \in \F^+$ with $I_{(d)} \neq \emptyset$, we have that $\a_{(d)}$ is injective since $\dom\a_{(d)} \subseteq \dom\a$. 
%Moreover $\a_{(d)}$ is a surjection since if $\a_{(d)}(i)=\a_{(d)}(j)$ for $i, j \in I$ then it must be that for some $\alpha, \beta <\kappa$, say $\alpha \leq \beta$, we have that $\a_\alpha(i+d)-\a_n(i)=\a_\beta(j+d)-\a_\beta(j)$, but this would imply that $i=j$ since $\a_\beta \leq \a_\alpha$ and of course $\a_\beta$ is surjective.

We will perform transfinite induction on $\kappa$ by meeting $\kappa$-many dense sets, one at a time. The fact that $\P$ is $<\kappa$-closed is what will carry us through the construction. Below, we establish 3 families of dense sets, each of size $\kappa$.

\begin{enumerate}
	\item \label{item:DomainDense} For each $i \in I$, the set $D_i=\set{\d \in \P}{i \in \dom \d }$ is dense. 
	
	To see this, let $\a \in \P$ with domain $A$ and $i \in I \setminus A$. We need to find $\d \in D_i$ satisfying $\d \leq \a$. In order to find such a $\d$, first we must ensure that the value we assign to $i$ is not equal to anything in the range of $\a$, namely $\d(i) \neq \a(a)$ for each $a \in A$. 
	
	Secondly, we must ensure the $T_d$-sequencings for $\d$ are injections. This amounts to ensuring that for each $a \in A_{(d)}$, 
		$$\a(a)^{-1}\a(a+d) \neq \d(i)^{-1}\a(i+d)$$ and/or
		$$\a(a)^{-1}\a(a+d) \neq \a(i-d)^{-1}\d(i)$$ if $i-d$ and/or $i+d$ happen to be in $A$. Since there are strictly less than $\kappa$ many elements in the range of $\a$, this leaves less than $\kappa$ many elements of $G$ to avoid assigning $\d(i)$, which is doable as $G$ has size $\kappa$, there are plenty of elements to choose from. 
	However, what of the case where $i-d$ and $i+d$ are both in $A$? We need to also make sure that $\d(i)^{-1}\a(i+d) \neq \a(i-d)^{-1}\d(i)$. As $G$ is abelian, this is the same as making sure that $$(\d(i))^2 \neq \a(i+d)\a(i-d).$$ Since $G$ is squareful, there are $\kappa$ many squares of elements of the group. So we have lots of choices for $(\d(i))^2$, and again, there are only less than $\kappa$ many possible elements of $G$ of the form $\a(i+d)\a(i-d)$, since the range of $\a$ has size less than $\kappa$.
	
	This means that ultimately the set of values to rule out for $\d(i)$ is at most size less than $\kappa$, and we just need to make sure it's not one of those values. As seen above, since $G$ is squareful and has size $\kappa$, this can be done. \\
	
	\item For each $g \in G$, the set $D_g=\set{\d \in \P}{g \in \ran \d }$ is dense. 
	
	Again, as in the above case, the idea should be that we only have to avoid less than $\kappa$ many scenarios, but we have room in $\R$ for that. 
	
	Suppose $g \in G \setminus \ran \a$. We need to find $\d \in D_g$ satisfying $\d \leq \a$. This amounts to finding $\overline g \notin A = \dom \a$ so that we can let $\d(\overline g)=g$, satisfying $\overline g \notin A_{(d)}$ whenever $A_{(d)}$ is nonempty for some $d \in \F^+$.
	
	Both $A$ and each $A_{(d)} \subseteq A$ have size less than $\kappa$, so this can be done. \\
		
	\item For each $g \in G$ and each $d \in \F^+$ with $I_{(d)}\neq \emptyset$, the set $D^d_g=\set{\d \in \P}{g \in \ran \d_{(d)}}$ is dense. 
	
	To see this, fix $d \in \F^+$ such that $I_{(d)}\neq \emptyset$ and let $g \in G$. Let $\a \in \P$, and suppose that $g \notin \ran\a_{(d)}$. We want to see that it is possible to extend $\a$ to a condition $\d \in D^d_g$ such that $g=\d(\overline g)^{-1}\d(\overline g+d)$ for some $\overline g \in I$. This amounts to finding a suitable $\overline g$. First we need $\overline g$ to be so that $\overline g \notin A_{(d)}$ where $A = \dom \a$. Then we need to ensure that $\d(\overline g), \d(\overline g+d) \notin \ran \a$, and also obviously that $g=\d(\overline g)^{-1}\d(\overline g+d)$. 

	It must also be the case that for any $a \in A$, we have that 
		$$\a(a)^{-1}\d(\overline g) \notin \ran \a_{(\overline g-a)}, \ \  \d(d)^{-1}\a(a) \notin \ran \a_{(a-\overline g)},$$ 
		$$\a(a)^{-1}\d(\overline g+d)  \notin \ran \a_{(\overline g+d-a)}, \ \ \d(\overline g+d)^{-1}\a(a) \notin \ran \a_{(a-\overline g-d)}.$$

Moreover, we can't inadvertently mess up another sequencing. In particular, whenever we have that both $\overline g+d', \overline g+d+d'\in A$ for some $d'\in I^+$, we must have that $$\d(\overline g)^{-1}\a(\overline g+d') \neq \d(\overline g+d)^{-1}\a(\overline g+d+d'),$$ meaning that, as $G$ is abelian, we must satisfy $$g= \d(\overline g)^{-1}\d(\overline g+d) \neq \a(\overline g+d')^{-1}\a(\overline g+d+d').$$ This contradicts the requirement that $g \notin \ran \a_{(d)}$. Dually, we need that whenever there is some $d'\in I^+$ such that $\overline g-d', \overline g-d-d' \in A$, we must have that $$\a(\overline g-d')^{-1}\d(\overline g) \neq \a(\overline g+d-d')^{-1}\d(\overline g+d),$$ but this again contradicts $g \notin \ran \a_{(d)}$.

Since we have only eliminated less than $\kappa$ many options, as we are restricted by $A$ and its image under $\a$, we have plenty of room to choose such an $\overline g$ as desired.\\
\end{enumerate}

Let $$\mathcal D = \set{D_i}{i\in I} \cup \set{D_g}{g\in G} \cup \set{D^d_g}{d \in \F^+\text{ with } I_{(d)}\neq \emptyset, g \in G},$$ and note that $|\mathcal D|=\kappa$, so we may enumerate all of the dense sets as $\mathcal D = \seq{\mathcal D_\alpha}{\alpha<\kappa}$. 
We define a function $\b$ by transfinite induction on $\kappa$. Start with any partial $T_\infty$-terrace on $\kappa$, say a finite one, for example, call it $\b_{-1}$.
The idea is to start meeting each of the dense sets in $\mathcal D$ one-by-one, in order to grow our partial $T_\infty$-terrace on $\kappa$. At stage $\alpha=0$ or when $\alpha \leq \omega$ is a successor ordinal, let $\b_\alpha \leq \b_{\alpha-1}$ satisfy $\b_\alpha \in \mathcal D_\alpha$. Density allows us to continue the construction from stage $0$ through all successor stages. At limit stages, say $\lambda <\kappa$, we use the fact that $\P$ is $<\kappa$-closed to find a condition strengthening the chain of our constructed $\b_\alpha$s for $\alpha<\gamma$ (the union of them will do), and then strengthen that condition to get inside the relevant dense set so that $\b_\lambda \in \mathcal D_\lambda$. 

By construction, $\cup \set{b_\alpha}{\alpha<\kappa}$ defines a function $\b: I \To G$ with the desired properties: 

\begin{enumerate}
	\item \emph{$\b$ is a bijection:} This is ensured by meeting, for each $i \in I$, the dense sets $D_i$ for injectivity and for meeting $D_g$ for each $g \in G$ for surjectivity.
	\item \emph{For each $d \in I^+$, $\b_{(d)}$ is a bijection:} The fact that the sequencing is injective is ensured by item \ref{item:DomainDense} as well, since at some point we will add both $i \in I$ and $i+d$ to the domain of the partial terrace we are constructing. The dense sets $D^d_g$ for each $g \in G$ guarantee surjectivity. \qedhere
\end{enumerate}
\end{proof}


\begin{cor}\label{cor:vatsquares}
For every index set~$I$ there is a Vatican square on~$I$.  In particular, there is a Vatican square of every infinite order.
\end{cor}

\begin{proof}
For every infinite order~$\kappa$ there is an abelian squareful group~$G$ of order~$\kappa$. Use~$G$ in Theorems~\ref{th:terrace2square} and~\ref{th:T_infty} to produce the required Vatican square.
\end{proof}

The question of which infinite groups admit directed $T_D$- and~$T_{\infty}$-terraces arises.  Vanden Eynden shows that all countably infinite groups have a directed 1-terrace on~$\N$ \cite{VE78} and the proof is easily adapted to apply to index set~$\Z$.  

The method of Theorem~\ref{th:T_infty} can be applied to an additional family of groups:

\begin{prop}\label{prop:allinv}
If every non-identity element of an infinite abelian group $G$ is an involution then $G$ has a directed~$T_{\infty}$-terrace for any index set of size~$|G|$. 
\end{prop}

\begin{proof}
\end{proof}



\section{Squares not based on groups}\label{sec:notgp}



The results of the previous section raise the question about what is and is not possible for infinite squares more generally.   Perhaps {\em all} countably infinite squares may be made complete, or even Vatican, with a suitable permutation of their rows and columns?   In a similar vein, it is known that all infinite Steiner triple systems are resolvable~\cite{DHW14}, an uncommon property among finite systems. However, Theorem~\ref{th:notrcls} scuppers this possibility, showing that for every index set (and hence every infinite order) there is a square that cannot be made row-complete via permuting columns.

In the other direction, a question asked (and answered positively) about finite squares was whether there exist row-complete Latin squares that are not based on groups, see \cite{CE91, DK15, Owens76}.  We answer the infinite version of this question, also positively, in Theorem~\ref{th:infvat}.  Indeed, this result gives a Vatican square that is not group-based for every index set.   All known finite Vatican squares are based on groups.

\begin{thm}\label{th:notrcls}
For every index set~$I$, there is a Latin square on~$I$ that cannot be made row-complete by permuting columns.
\end{thm}

\begin{proof}
We build a Latin square on $\kappa=|I|$, which gives us one on $I$ via any bijection between $I$ and $\kappa$, with symbol set $\kappa$ as well. We will proceed as in the previous constructions by transfinite induction on $\kappa$. 

Consider the poset $\P$ consisting of partial Latin squares on $\kappa$, in a sense rectangles; more precisely injective partial functions from $\kappa\times\kappa\To\kappa$ whose domains are initial segments of $\kappa\times\kappa$ of size strictly less than $\kappa$. We think of these as rectangles, with ordinals less than $\kappa$ arranged in a matrix/grid on $\kappa$, rather than as functions. These rectangles must satisfy that each ordinal appears at most once in each row/column (they are Latin). Moreover we require that the for a Latin rectangle to be in the poset it is what we will call \textit{immune}, meaning it can't be made row-complete by permuting the columns. Partially order the immune Latin rectangles by $l\leq m$ if and only if $m$ is an initial segment of $m$.

We proceed by transfinite induction on $\kappa$, by first starting with the following immune Latin rectangle on $\kappa$, an element of $\P$, which we name $l_{-1}$:
$$
\begin{array}{ccc}
2  & 0 & 1 \\ 
1 & 2 &  0  \\
 0  & 1 & 2 
\end{array}
$$
We then begin meeting $\kappa$-many dense sets one by one, creating a chain of rectangles $l_\alpha$ for $\alpha<\kappa$ that each extend the previous ones. In order to carry ourselves through limit stages, we use the fact that $\P$ is $<\kappa$-closed. Indeed, given a chain of partial Latin rectangles, the union of all of them will also be a partial Latin rectangle - since any obstruction in the union would have to have shown up in one of the elements of the chain.

Let's define the desired dense sets to meet and show that they are dense.
\begin{enumerate}
	\item For each $\alpha, \beta<\kappa$, the set $D_{\alpha,\beta}=\set{l \in \P}{\alpha \text{ appears in row } \beta \text{ of } l}$ is dense.
	
	In other words, we need to see how to add a desired number $\alpha$ to row $\beta$ of a condition $l$, if it isn't already there. 

	If row $\beta$ is not already in the rectangle $l$, then we simply add rows with whatever values in them that we would like, without violating the Latin constraints, until reaching row $\beta$ in which we make sure to include $\alpha$ in it. Potentially add new columns if it is impossible not to in order to have $\alpha$ in row $\beta$. We thus arrive at a larger Latin rectangle $l'$ which contains $l$, but may not immune.
	Otherwise we would like to add an entry to an already existing row in $l$. Add a column with the desired entry, $\alpha$, in the desired row $\beta$, and put whatever you want for the other entries of the column, without violating the Latin constraints, to produce a new rectangle $l'$.
	
	In either case we have produced a larger Latin rectangle $l'$ which contains $l$ but may not be immune. In either case we should then perform the immunization procedure to make sure that we produce an immune Latin rectangle.
\begin{description}
		\item[Immunization:] For each combination of three different columns from the columns in our new rectangle $l'$, pick the least number $\gamma$ not appearing in $l'$, and add the following three columns on top of the three different columns: 
				$$\begin{array}{ccccc}
			\gamma+2  && \gamma   &&  \gamma+1 \\ 
			\gamma+1  && \gamma+2 &&  \gamma \\
			\gamma  && \gamma+1 && \gamma+2
		\end{array}$$ in whatever order you would like. Keep doing this with every possible combination of three different columns until all of the possible combinations of 3 different columns have been exhausted. Then fill in the gaps with whatever you would like without contravening the Latin constraints. By design, this produces a new condition $l''$ extending $l'$ but in the dense set $D_{\alpha, \beta}$.
		\end{description}
		To illustrate the technique, let's say that we would like to add the number $3$ to row $0$ in our initial immune Latin rectangle which we called $l_{-1}$. Before following the immunization procedure, first we produce the following rectangle $l_{-1}'$ 
		$$\begin{array}{cccc}
			2  & 0   &  1 & \color{ForestGreen} 5 \\ 
			1 & 2 &  0  & \color{ForestGreen} 4\\
			 0  & 1 & 2 & \color{red} 3
		\end{array}.$$
		We added 3 to the first row of a new column, and then filled up the column with the minimum values possible while making sure it is Latin. In this case the immunization isn't required, but we'll go ahead and do it to illustrate the process, and ensure it. In this example, there are 4 possible combinations of 3 different columns, since we have 4 columns at this point, so we end up adding 12 new rows to the top of our condition.		
		$$\begin{array}{cccc}
						&\color{blue}17& \color{blue}15&\color{blue}16\\
						&\color{blue}16&\color{blue}17&\color{blue}15\\
						&\color{blue}15&\color{blue}16&\color{blue}17\\
						\color{cyan}14&&\color{cyan}12&\color{cyan}13\\
						\color{cyan}13&&\color{cyan}14&\color{cyan}12\\
						\color{cyan}12&&\color{cyan}13&\color{cyan}14\\
						\color{blue}11&\color{blue}9&&\color{blue}10\\
						\color{blue}10&\color{blue}11&&\color{blue}9\\
						\color{blue}9&\color{blue}10&&\color{blue}11\\
						\color{cyan}8&\color{cyan}6&\color{cyan}7&\\
						\color{cyan}7&\color{cyan}8&\color{cyan}6&\\
						\color{cyan}6&\color{cyan}7&\color{cyan}8&\\
						2  & 0   &  1 &\color{ForestGreen}5\\ 
						1 & 2 &  0 & \color{ForestGreen}4 \\
						0  & 1 & 2 & {\textcolor{red} 3}
		\end{array}$$
Then we can fill in the above with whatever values we would like, so long as the rectangle $l_{-1}''$ we end up with is Latin.
	
	
		\item For each $\alpha, \beta<\kappa$, the set $D^{\alpha, \beta}=\set{l\in \P}{\alpha \text{ appears in column } \beta \text{ of } l}$ is dense. 
		
		Here we need to add a specified number $\alpha$ to the column number $\beta$ of a condition $l$, if it isn't already there. If $l$ already has at least $\beta$-many columns, then we can just add a new row to the top of $l$ to produce a new condition in the dense set extending $l$; with the desired value $\alpha$ in the desired column position, and anything else we want that doesn't contravene the Latin constraints. 
		
		If the condition has less than $\beta$-many columns, add as many columns as necessary, placing whatever values we want while ensuring the rectangle we produce is Latin. Once we get to the $\beta$th column we then put $\alpha$ somewhere in that column (potentially adding a new row to do this if we've already put $\alpha$ in all of the existing rows) producing the Latin rectangle $l'$ extending $l$. We've produced a Latin rectangle with the right value in the right position. 
To ensure that the new condition we produce is row complete under permuting columns, run through the same exact immunization procedure as outlined above, to produce a new condition which extends $l'$ (and thus also $l$) in the dense set as required.
\end{enumerate}

We enumerate the $\kappa$-many dense sets and meet them one-by-one as described, using the fact that $\P$ is $<\kappa$-closed to pass through limit stages. Then we produce $L=\cup \set{l_\alpha}{\alpha<\kappa}$, which defines a Latin square with the desired properties. Indeed, it isn't possible to permute columns in $L$ to obtain a row-complete square, since then it would be possible to permute the columns in some large enough $l_\alpha$, but each $l_\alpha$ is immune so that can't happen. Similarly, $L$ is Latin since each $l_\alpha$ is. Moreover, by meeting the dense sets we've ensured that the square is total, every column and row contains all ordinals below $\kappa$.
\end{proof}



Prior to giving Theorem~\ref{th:infvat} we need a result that lets us be sure that a square is not group-based.  The {\em quadrangle criterion} states that in a square based on a group if the three equations
$$a_{i_1j_1} = a_{i_2j_2}, \ a_{k_1j_1} = a_{k_2j_2}, \ a_{i_1l_1} = a_{i_2l_2}$$
are satisfied then $a_{k_1l_1} = a_{k_2l_2}$ \cite[Theorem~1.2.1]{DK15}.  That is, if two ``quadrangles" in a group-based square agree on three points then they agree on the fourth.


\begin{thm}\label{th:infvat}
For every index set~$I$ there is a Vatican square on~$I$ that is not based on a  group.
\end{thm}

\begin{proof}
Let $\F_\kappa$ be an ordered field on $\kappa=|I|$.
We build a Vatican square $L$ on $\F_\kappa$ with symbol set $\kappa$ by transfinite induction. The strategy to build one on an arbitrary index set is exactly the same as outlined below, potentially starting with a different starting sub-square we call $p_{-1}$ depending on your index set and symbol set.

Again we find it useful to define an appropriate poset, to build the Vatican square by growing a finite one via meeting dense sets. The conditions in our poset $\P$ consist of partial Vatican squares which are not group based. These are injective partial functions of the form $p:\F_\kappa \times \F_\kappa\To\kappa$ whose domains are initial segments of $\F_\kappa \times \F_\kappa$ of size strictly less than $\kappa$. Moreover, these squares should be Vatican in the sense of a finite square, in that for each $d\leq \F_\kappa^+$ such that ${\F_\kappa}_{(d)}\neq\emptyset$ we have that each ordered pair of distinct symbols coming from $\kappa$ appear at distance $d$ at most once in each order in rows and at most once in each order in columns. We partially order $\P$ by $p \leq q$ so long as rows and columns of $q$ form initial segments of rows and columns of $p$ respectively.

Clearly $\P$ is $<\kappa$-closed.

We start the induction with the following partial Vatican square on $\kappa$, call it $p_{-1}$:
$$\begin{array}{cccc}
	\color{blue} 0 & 5 & 6 & \color{blue} 1 \\ 
	7 & \color{blue} 0 & \color{blue}1 & 8  \\
	9 & \color{blue}2 & \color{blue} 3 & 10 \\
	\color{blue} 2 & 11 & 12 & \color{blue} 4 
\end{array}$$
No square containing this as a subsquare can be group-based since it fails to satisfy the quadrangle criterion.
We then begin meeting dense sets in some enumeration of the following four families of dense sets. We successively extend $p_{-1}$ and each other, with a chain of $p_\alpha$'s, in $\kappa$ many stages, using the $<\kappa$-closure of $\P$ to get through limit stages. 

\begin{enumerate}
	\item For each $\alpha, \beta <\kappa$, the set $D_{\alpha,\beta}=\set{p \in \P}{\alpha \text{ appears in row } \beta \text{ of } p}$ is dense.
	
	The trick here is not to mess up the Vatican property on the square. If $\alpha$ does not appear in row $\beta$ of a condition $p$ yet, it is not necessarily enough to simply add it to the end of row $\beta$, since it is possible that then $\alpha$ appears more than once at some distance from another element in that row or column. We are only guaranteed to be safe with that method if $\alpha$ doesn't already appear anywhere in the partial square. Thus the idea is to fill up the necessary entries of the square with new symbols that haven't appeared anywhere in the partial square yet. Since the partial Vatican squares in $\P$ are bounded in each row by $\kappa$, we know that there are many entries above the maximum value of an entry already in each row and column. We can keep adding such entries until we are safe to add $\alpha$ in row $\beta$. 
	
	If $p$ doesn't even have a row $\beta$ yet, then add new rows which have the same length as the longest row in $p$, with symbols not already in $p$. On the $\beta$th row, do the same thing, but add $\alpha$ at the end, creating a new column with just $\alpha$ in it.
	
	Otherwise, $p$ already has entries in row $\beta$, this can again at worst amount to essentially doubling $p$ in size, by adding enough new elements (at worst the length of its longest row plus $\beta$) to row $\beta$ so that no column or row distance between an element of row $\beta$ of $p$ and $\alpha$ could even have occurred in $p$ initially.
	
	For example, with the above square $p_{-1}$, if we would like to add 0 to the bottom row:
	$$\begin{array}{cccccccc}
	0 & 5 & 6 & 1 \\ 
	7 &  0 & 1 & 8  \\
	9 & 2 & 3 & 10 \\
	2 & 11 & 12 & 4 & \color{ForestGreen} 13 & \color{ForestGreen} 14 & \color{ForestGreen}15 & \color{red}0
\end{array}$$
and if we would like to add 0 to the sixth row:
$$\begin{array}{ccccccc}
	\color{ForestGreen}17 & \color{ForestGreen}18 & \color{ForestGreen}19 & \color{ForestGreen}20 & \color{red} 0\\
	\color{ForestGreen}13 &  \color{ForestGreen}14 & \color{ForestGreen}15 & \color{ForestGreen}16\\
	0 & 5 & 6 & 1 \\ 
	7 &  0 & 1 & 8 \\
	9 & 2 & 3 & 10 \\
	2 & 11 & 12 & 4
\end{array}$$

	
	\item For each $\alpha, \beta <\kappa$, the set $D^{\alpha,\beta}=\set{p \in \P}{\alpha \text{ appears in column } \beta \text{ of } p}$ is dense.
	
	Same procedure as with rows.
	
	\item For each $\alpha, \beta<\kappa$ and $d \in \F_\kappa^+$, the set $$E^d_{\alpha, \beta}=\set{p \in \P}{\text{$\alpha$, $\beta$ appear at distance $d$ apart in some row}}$$ is dense.
	
	If $\alpha$ and $\beta$ already do not appear distance $d$ apart anywhere in a partial Vatican square $p$, what we should do to absolutely guarantee we have no conflicts is start a new row. Add new entries to this row with symbols different than all of the symbols appearing in $p$ so far, up to the length of the longest row. Then add $\alpha$, followed by $d-1$ many symbols, and then $\beta$. This indeed will still be a condition, since $\beta$ and $d$ are less than $\kappa$.
	
	For example, with our starting square $p_{-1}$, this is how we would extend it to have 0 and 1 at distance 2 apart:
$$\begin{array}{ccccccc}
	\color{ForestGreen} 13 & \color{ForestGreen}14 & \color{ForestGreen}15 & \color{ForestGreen}16 & \color{red}0 &\color{ForestGreen}17 &\color{red}1\\
	0 & 5 & 6 & 1 \\ 
	7 &  0 & 1 & 8  \\
	9 & 2 & 3 & 10 \\
	2 & 11 & 12 & 4 
\end{array}$$
	
	\item For each $\alpha, \beta<\kappa$ and $d \in \F_\kappa^+$, the set $$E_d^{\alpha, \beta, \delta}=\set{p \in \P}{\text{$\alpha$, $\beta$ appear at distance $d$ apart in some column}}$$ is dense.
	
	Same procedure as with rows.
\end{enumerate}

We produce a square $L$ at the end of this construction by taking the union of all of the $p_\alpha$'s in the chain we described building above. Clearly $L$ is Latin, since at some stage every ordinal less than $\kappa$ was added to every row and every column as guaranteed by our first two families of dense sets. It must be that $L$ is Vatican as well. First of all, we know that the pair occurrence for each row and column must be satisfied at least once in each row and column by meeting the last two families of dense sets described above. Moreover if this happened somewhere more than once, it would have to happen in some condition $p_\alpha$, but conditions in $\P$ are not allowed to have this property.
\end{proof}


\section{Semi-Vatican squares}\label{sec:semivat}

Infinite Semi-Vatican squares---recall that these are squares in which each pair of symbols appears exactly once at each distance~$d$ in rows and columns, rather than exactly once in each order---behave very similarly to Vatican squares.  

The required generalization of~directed $T_{D}$- and~$T_{\infty}$-terraces are directed $S_{D}$- and~$S_{\infty}$-terraces.  As before, let~$I$ be an index set in an ordered field~$\F$.  Let~$G$ be a group of order~$|I|$.  For a bijection~${\bf a}: I \rightarrow G$ define a function~${\bf a}_{(d)}: I \rightarrow G \setminus \{ e \}$ for each~$d \in \F^+$ with~$I_{(d)} \neq \emptyset$ by
$${\bf a}_{(d)}(i) = {\bf a}(i)^{-1}{\bf a}(i+d).$$
If~$G$ has no involutions, and if there is a~$D$ such that for all~$d < D$ with~$I_{(d)} \neq \emptyset$ we have that the image of ${\bf a}_{(d)}$ contains exactly one occurrence from each set~$\{ x,x^{-1} : x \in G\setminus \{e\} \}$, then~${\bf a}$ is a {\em directed $S_D$-terrace for~$G$}.  If ${\bf a}_{(d)}$ has this property for all~$d$ with~$I_{(d)} \neq \emptyset$ then ${\bf a}$ is  a {\em directed $S_{\infty}$-terrace} for~$G$.

The requirement that~$G$ has no involutions comes into play when we consider constructing semi-Vatican squares using the method of Theorem~\ref{th:terrace2square}.  Suppose~$z \in G$ is an involution and~${\bf a}_{(d)} = z$ for some bijection~${\bf a}$ with the usual definition for~${\bf a}_{(d)}$.  If~${\bf a}(i) = x$, then ${\bf a}(i+d) = xz$.  There is a~$j$ such that~${\bf a}(j) = xz$ and now~${\bf a}(i+d) = xz^2 = x$.  Thus the pair~$\{ x, xz \}$ occurs twice at distance~$d$ in~$L({\bf a})$, once in each order.  Hence a square constructed with this method using a group with an involution cannot be semi-Vatican.



The proofs of the previous two sections require only minor modifications to give the following slate of results:

\begin{thm}\label{th:semi_terrace2square}
Let~$G$ be a group of infinite order~$\kappa$ with no involutions.  If~$G$ has a directed $S_{\infty}$-terrace for an index set~$I$ then there is a semi-Vatican square of order~$|G|$.
\end{thm}

\begin{thm}\label{th:semi_T_infty}
Let~$G$ be an involution-free abelian squareful group of infinite order~$\kappa$.   Then~$G$ has a directed $S_{\infty}$-terrace.
\end{thm}

\begin{cor}\label{cor:semi_vatsquares}
For every index set~$I$ there is a semi-Vatican square on~$I$.  In particular, there is a semi-Vatican square of every infinite order.
\end{cor}

\begin{thm}\label{th:semi_infvat}
For every index set~$I$ there is a semi-Vatican square on~$I$ that is not based on a  group.
\end{thm}

All of the existence results presented so far are non-constructive and rely on transfinite induction.  Perhaps surprisingly, in the case when the group is~$(\R, +)$, the tools of undergraduate calculus are sufficient to construct to a semi-Vatican square.


\begin{thm}\label{th:svr}
There is a semi-Vatican square on index set~$\R$ based on~$(\R,+)$.
\end{thm} 

\noindent
Proof.   We give a direct definition for a directed $S_{\infty}$-terrace~${\bf a}$:
\begin{equation*}
    {\bf a}(x) = \begin{cases}
               e^x   -1            & x \geq 0\\
               -\ln (1-x)       & \text{otherwise}
           \end{cases}
\end{equation*}
This is a continuous, strictly increasing bijection from~$\R$ to~$\R$.  Its derivative is:
\begin{equation*}
    {\bf a}'(x) = \begin{cases}
               e^x               & x \geq 0\\
              \frac{1}{1-x}       & \text{otherwise}
           \end{cases}
\end{equation*}
which is a continuous, strictly increasing bijection from~$\R^+$ to~$\R^+$.

Therefore, for each~$d \in \R^+$, we have that ${\bf a_{(d)}}$ is a bijection from~$\R^+$ to~$\R^+$.  Hence~${\bf a}$ is a directed $S_{\infty}$-terrace and Theorem~\ref{th:semi_terrace2square} gives a semi-Vatican square based on~$I = \R$.
\qed

Similar approaches for Vatican squares quickly run into difficulties.



\section{Orthogonality}\label{sec:orth}

Let~$G$ be a group and~$\theta: G \rightarrow G$ a bijection.  If $g \mapsto g^{-1}\theta(g)$ is a bijection then~$\theta$ is an {\em orthomorphism}; if  $g \mapsto g\theta(g)$ is a bijection then~$\theta$ is a {\em complete mapping}.  Two orthomorphisms, $\theta, \phi$ are {\em orthogonal} if $g \mapsto \theta(g)^{-1} \phi(g)$ is a bijection.

\texttt{[orthomorphisms $\rightarrow$ orthogonal latin squares.]}

It's known that every infinite group has an orthomorphism~\cite{Bateman50}, so there is a pair of orthogonal Latin squares at every infinite order.


The work of Section~\ref{sec:cayley} may be adapted to give families of mutually orthogonal orthomorphisms. \texttt{[$R_{\infty}$ stuff.]}



\begin{thm}
Let~$G$ be a group of infinite order~$\kappa$.  If~$G$ has a directed $R_{\infty}$-terrace then~$G$ has a set of~$ \kappa$ mutually orthogonal orthomorphisms. 
\end{thm}

\begin{proof}
Let~${\bf a}$ be a directed $T_{\infty}$-terrace for~$G$ over some index set~$I$.  For each~$d$ such that~$I_d \neq \emptyset$ (of which there are~$\kappa$) define~$\theta_d(e) = e$ and $\theta_d( {\bf a}(i) ) = {\bf a}(i+d)$.  This gives us the orthogonal orthomorphisms we're looking for:

First, they are orthomorphisms: given~$g \in G \setminus \{ e \}$ with~${\bf a}(i) = g$ we get
$$g^{-1}\theta_d(g) = {\bf a}(i)^{-1}{\bf a}(i+d) = {\bf a}_{(d)}(i)$$
which, when we also consider that~$\theta_d(e)=e$, gives us a bijection on~$G$.

Second, they are orthogonal: again taking~$g \in G \setminus \{ e \}$ with~${\bf a}(i) = g$, if~$d_2 > d_1$ we get: 
$$\theta_{d_1}^{-1}(g) \theta_{d_2}(g) =  {\bf a}(i+d_1)^{-1}{\bf a}(i+d_2)  = {\bf a}_{(d_2 - d_1)}(i+d_1) .       $$
If~$d_1 < d_2$ we get:
$$\theta_{d_1}^{-1}(g) \theta_{d_2}(g) =  {\bf a}(i+d_1)^{-1}{\bf a}(i+d_2)  = {\bf a}_{(d_1 - d_2)}(i+d_2)^{-1} .       $$
Also~$\theta_{d_1}(e)^{-1}\theta_{d_2}(e) =e$ in each case, giving bijections on~$G$.
\end{proof}

If we have a directed~$R_D$-terrace, with the obvious definition, then the same argument gives $ | \{ d \leq D : I_d \neq \emptyset \} |$ mutually orthogonal orthomorphisms for~$G$.

\begin{thm}\label{th:R_infty}
Let~$G$ be an abelian squareful group of infinite order.   Then~$G$ has a directed $R_{\infty}$-terrace.
\end{thm}

\begin{proof}
The only difference between a directed~$R_{\infty}$-terrace and a directed $T_{\infty}$-terrace is that the identity is not in the domain of a directed~$R_{\infty}$-terrace.  The presence of the identity is not relied upon in the Theorem~\ref{th:T_infty}'s proof that abelian squareful groups of infinite order have a  directed $T_{\infty}$ terrace; a simple adjustment of the argument produces the required directed~$R_{\infty}$-terrace.
\end{proof}

This immediately gives:

\begin{cor}
There is a set of~$\kappa$ mututally orthogonal squares of order~$\kappa$ for all infinite cardinalities~$\kappa$.
\end{cor}


\texttt{[Prove stronger result about orthomorphisms that does not go via directed $R_{\infty}$-terraces?]}





Another embellishment of the complete mapping concept is the strong complete mapping:
If~$\theta$ is both an orthomorphism and a complete mapping then it is a {\em strong complete mapping}.  Every countably infinite group has a strong complete mapping~\cite{Evans12}.

\texttt{[Prove that all infinite groups have strong complete mappings?]}





\section*{Acknowledgements}

This work was partly funded by a Marlboro College Faculty Professional Development Grant and a Marlboro College Town Meeting Scholarship Fund award.  The authors are grateful for this assistance.


\begin{thebibliography}{99}


\bibitem{Anderson90} 
B.~A.~Anderson, Some quasi-2-complete latin squares, {\em Congr. Numer.} {\bf 70} (1990) 65--79.

%\bibitem{AandI92}
%B.~A.~Anderson and E.~C.~Ihrig.
%\newblock All groups of odd order have starter-translate 2-sequencings,
%\newblock {\em Australas. J. Combin.} {\bf 6} (1992) 135--146.

%\bibitem{AandI93} B.~A.~Anderson and E.~C.~Ihrig,
%Symmetric sequencings of non-solvable groups,
%{\em Congr. Numer.} {\bf 93} (1993) 73--82.


%\bibitem{Bailey84}
%R.~A.~Bailey,
%Quasi-complete {L}atin squares: construction and randomization,
%{\em J.~Royal Statist. Soc. Ser. B} {\bf 46} (1984) 323--334.



\bibitem{Bateman50}
P.~T.~Bateman, A remark on infinite groups, {\em Amer.~Math.~Monthly} {\bf 57} (1950), 623--624.

\bibitem{BM91}
J.~V.~Brawley and G.~L.~Mullen, Infinite latin squares containing nested sets of mutually orthogonal finite latin squares, {\em Publicationes Mathematicae-Debrecen} {\bf 39} (1991) 135--141.


\bibitem{CW02}
P.~J.~Cameron and B.~S.~Webb, What is an infinite design? {\em J.~Combin.~Des.} {\bf 10} (2002) 79--91.

\bibitem{Caulfield96}
M.~J.~Caulfield, Full and quarter plane complete infinite Latin squares, {\em Discrete Math.} {\bf 159} (1996) 251--253.

\bibitem{TuscanCRC}
W.~Chu, S.~W.~Golomb and H.-Y.~Song, Tuscan Squares, in {\em The Handbook of Combinatorial Designs (2nd Edition), Eds. C.~J.~Colbourn and J.~H.~Dinitz}, Chapman and Hall/CRC, (2007).

\bibitem{CE91}
D.~Cohen and T.~Etzion, Row complete Latin squares that are not column complete, {\em Ars Combin.}~{\bf 32} (1991) 193--201.

\bibitem{DHW14}
P.~Danziger, D.~Horsley and B.~S.~Webb, Resolvability of infinite designs, {\em J.~Combin.~Thy.~A} {\bf 123} (2014) 73--85.

\bibitem{DK15}
J.~D{\'e}nes and A.~D.~Keedwell, {\em Latin Squares and Their Applications (2nd Ed.)}, Elsevier (2015).

\bibitem{EGT89}
T.~Etzion, S.~W.~Golomb, and H.~Taylor, Tuscan-$K$ squares, {\em Advances in Applied Math.} {\bf 10} (1989) 164--174.


\bibitem{Evans12}
A.~B.~Evans, The existence of strong complete mappings, {\em Electronic J.~Combin.}~{\bf 19} (2012), \#P34.

%\bibitem{Gallian}
%J.~A.~Gallian, A dynamic survey of graph labellings, Electron. J. Combin. {\bf DS6} (2001--2010), 246pp.

%\bibitem{gap}
%{{G}{A}{P} group}.
%\newblock {G}{A}{P}---{G}roups, {A}lgorithms, and {P}rogramming, {V}ersion 4, 1999.

\bibitem{Gordon61} B.~Gordon, Sequences in groups with distinct partial products, {\em Pacific J. Math.} {\bf 11} (1961) 1309--1313.

\bibitem{Higham98} 
J.~Higham, Row-complete Latin squares of every composite order exist, {\em J. Combin. Des.} {\bf 6} (1998) 63--77. 

\bibitem{HW05}
A.~J.~W.~Hilton and J.~Wojciechowski, Amalgamating infinite Latin squares, {\em Discrete Math.} {\bf 292} (2005) 67--81.

%\bibitem{HLW}
%Y.-S.~Hwang, D.~B.~Leep and A.~R.~Wadsworth,
%Galois groups of order $2n$ that contain a cyclic subgroup of order $n$, {\em Pacific J. Math.} {\bf 212} (2003) 297--319.

%\bibitem{Lucas92}
%\'{E}.~Lucas, {\em R\'{e}cr\'{e}ations Math\'{e}mathiques}, T\^{o}me II, Albert Blanchard, Paris, 1892 (reprinted 1975).

%\bibitem{survey}
%M.~A. Ollis, Sequenceable groups and related topics, {\em Electron. J. Combin.} {\bf DS10} (2002) 34pp.

%\bibitem{Ollis05}
%M.~A.~Ollis, On terraces for abelian groups, {\em Disc. Math.} {\bf 305} (2005) 250--263.

%\bibitem{Ollis12}
%M.~A.~Ollis,  A note on terraces for abelian groups,  {\em Australas.~J.~Combin.}  {\bf 52} (2012),  229--234.

%\bibitem{OW1}
%M.~A.~Ollis and D.~T.~Willmott, On twizzler, zigzag and graceful terraces, {\em Australas.~J.~Combin.} {\bf 51} (2011) 243--257.

\bibitem{Ollis14}
M.~A.~Ollis, New complete Latin squares of odd order, {\em Europ.~J.~Combin.}~{\bf 41} (2014) 35--46.

\bibitem{OllisTFSG}
M.~A.~Ollis, Terraces for small groups, {\em submitted}, \texttt{arXiv:1603.01496}.

\bibitem{Owens76}
P.~J.~Owens,  Solutions to two problems of Denes and Keedwell on row-complete
Latin squares, {\em J.~Combin.~Thy.}~A {\bf 21} (1976) 299--308.

%\bibitem{PreeceZF}
%D.~A.~Preece, Zigzag and foxtrot terraces for $\Z_n$, {\em Australas. J. Combin.} {\bf 42} (2008) 261--278.

%\bibitem{Scott64}
%W.~R.~Scott, {\em Group Theory}, Prentice-Hall, New Jersey (1964).


\bibitem{VE78}
C.~Vanden Eynden, Countable sequenceable groups, {\em Discrete Math.} {\bf 23} (1978) 317--318.


%\bibitem{Williams49}
%E.~J. Williams,
%Experimental designs balanced for the estimation of residual effects of treatments,
%{\em Aust. J. Scient. Res. A}, {\bf 2} (1949) 149--168.






\end{thebibliography}

\end{document}
